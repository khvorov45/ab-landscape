\documentclass[12pt]{article}

\usepackage{float}

\usepackage{standalone}

%\usepackage[utf8x]{inputenc}

%%% PAGE DIMENSIONS
\usepackage{geometry}
\geometry{a4paper}
\geometry{margin=2.54cm} % for example, change the margins to 2 inches all round

\usepackage{graphicx} % support the \includegraphics command and options

\usepackage[parfill]{parskip} % Activate to begin paragraphs with an empty line rather than an indent

%%% PACKAGES
\usepackage{booktabs} % for much better looking tables
\usepackage{array} % for better arrays (e.g., matrices) in maths
\usepackage{paralist} % very flexible & customisable lists (e.g., enumerate/itemise, etc.)
\usepackage{verbatim} % adds environment for commenting out blocks of text & for better verbatim
\usepackage{subfig} % make it possible to include more than one captioned figure/table in a single float
% These packages are all incorporated in the memoir class to one degree or another...

\usepackage{multicol}
\usepackage{multirow}
\usepackage{xcolor}
\usepackage{amsmath}

\usepackage[T1]{fontenc}
\usepackage{lmodern}

% Sans-serif font
\renewcommand{\familydefault}{\sfdefault}

\usepackage{makecell}

\renewcommand{\arraystretch}{1.1}

%%% HEADERS & FOOTERS
\usepackage{fancyhdr} % This should be set AFTER setting up the page geometry
\pagestyle{fancy} % options: empty , plain , fancy
\fancyhead[L]{\leftmark}
\fancyhead[C]{}
\fancyhead[R]{\rightmark}
\fancyfoot[L]{}
\fancyfoot[C]{}
\fancyfoot[R]{\thepage}
\renewcommand{\headrulewidth}{0pt}
\renewcommand{\footrulewidth}{0pt}
\setlength{\headheight}{52pt}

\fancypagestyle{plain}{
	\fancyhf{} % clear all header and footer fields
	\fancyfoot[R]{\thepage} % except the center
	\renewcommand{\headrulewidth}{0pt}
	\renewcommand{\footrulewidth}{0pt}
}

%%% BIBLIOGRAPHY
\usepackage[numbers]{natbib}
\bibliographystyle{vancouver}

%%% SECTION TITLE APPEARANCE
\usepackage{sectsty}
\allsectionsfont{\sffamily\mdseries\upshape}

%%% ToC (table of contents) APPEARANCE
%\usepackage[nottoc,notlof,notlot]{tocbibind} % Put the bibliography in the ToC
%\usepackage[titles,subfigure]{tocloft} % Alter the style of the Table of Contents
%\renewcommand{\cftsecfont}{\rmfamily\mdseries\upshape}
%\renewcommand{\cftsecpagefont}{\rmfamily\mdseries\upshape} % No bold!

\usepackage[bookmarks,bookmarksnumbered,bookmarksopen,hidelinks]{hyperref}

\usepackage{bookmark}

% TITLE
\title{CDC}
\author{Arseniy Khvorov}
\begin{document}

\maketitle

\section{Viruses}

The viruses and their clades are in Table \ref{tab:cdc-clade-viruses}. The frequencies of clades throughout the study period are in Figure \ref{fig:cdc-clade-freq} (data from Nextstrain). I assume these global frequencies to be representative of what they were at the sites where the data was collected (Israel and Peru).

\input{../data-summary/cdc-clade-viruses.tex}

\begin{figure}
	\includegraphics[width=\textwidth,height=\textheight,keepaspectratio]{../data-summary/cdc-clade-freq.pdf}
	\caption{Clade frequencies from 2016 to 2019. Two sets of frequencies per year. Data from Nexstrain. Whenever they do not add up to 100\%, it means that the remaining circulating flu was made up of clades whose viruses we do not have titre data against (e.g., clade 'A3').}
	\label{fig:cdc-clade-freq}
\end{figure}

\section{Confidence intervals}

All confidence intervals (e.g. on GMTs) are normal-approximation confidence intervals for the mean. In case of GMTs the "mean" is the mean of the log-titres. In case of titre changes between timepoints the "mean" is the mean of log-titre differences between timepoints.

\section{Objective 1}

Compare post-vaccination HI antibody titres of groups with different vaccination histories.

Hypothesis: post-vaccination titres and the pre/post vaccination titre rise
is lower in the frequently vaccinated.

\subsection{Data}

The data comes from Israel and Peru sites over three years. Each participant was followed up for one year. Vaccination history was collected. A measure of prior vaccination history I will use is the number of prior vaccinations in the 5 years before recruitment (e.g., 2011-2015 for participants recruited in 2016).

Table \ref{tab:cdc-participant-summary-obj1} contains sex and age summaries for participants split by the number of prior vaccinations, site and year of recruitment.

\input{../data-summary/cdc-participant-summary-obj1.tex}

\subsection{HI titres}

Geometric mean titres for each virus at each timepoint for each vaccination history group (0, 1-3 or 5 prior vaccinations) split by site are shown in Figure
\ref{fig:cdc-obj1-timepoint-gmts}.
Pre-vaccination GMTs are generally close to the same
or higher for all viruses for the vaccinated groups as compared to unvaccinated group.
Post-vaccination GMTs are generally close to the same or higher in the unvaccinated compared to the other groups. The pattern of GMTs for the post-season timepoint is close to that
of post-vaccination except that titres are generally more similar between the groups.

\begin{figure}
	\includegraphics[width=\textwidth,height=\textheight,keepaspectratio]{../data-summary/cdc-obj1-timepoint-gmts.pdf}
	\caption{Objective 1 geometric mean titres and 95\% confidence intervals for each virus at each timepoint for the group with 0, 1-3 and 5 vaccinations in the 5 years before the bleed. The shaded regions indicate the vaccine strains (study year 1 and 2 - A/Hong Kong/4801/14e, study year 3 - A/Singapore/16-0019/16e).}
	\label{fig:cdc-obj1-timepoint-gmts}
\end{figure}

\subsection{HI titre changes}

Mean ratio of post-vaccination titres to pre-vaccination titres split by site is shown in Figure
\ref{fig:cdc-obj1-timepoint-diffs}. Ratios are higher for the infrequently
vaccinated for the majority of viruses.

\begin{figure}
	\includegraphics[width=\textwidth,height=\textheight,keepaspectratio]{../data-summary/cdc-obj1-timepoint-diffs.pdf}
	\caption{Objective 1 ratios of post-vaccination titres to pre-vaccination titres and 95\% confidence intervals for each virus for the frequently and the infrequently vaccinated. The shaded regions indicate the vaccine strains (study year 1 and 2 - A/Hong Kong/4801/14e, study year 3 - A/Singapore/16-0019/16e).}
	\label{fig:cdc-obj1-timepoint-diffs}
\end{figure}

\subsection{GMT against circulating strains}

The list of viruses for each clade is in Table \ref{tab:cdc-clade-viruses}. Frequencies of different virus clades during the study period are in Figure \ref{fig:cdc-clade-freq}. The timing of sample collection is in Figure \ref{fig:cdc-obj1-bleed-dates}. I used the frequencies for the first half of the corresponding years for Peru and the second half for Israel.

\begin{figure}
	\includegraphics[width=\textwidth,height=\textheight,keepaspectratio]{../data-summary/cdc-obj1-bleed-dates.pdf}
	\caption{Bleed dates for objective 1 data. Each horizontal line represents a participant. Points are bleeds.}
	\label{fig:cdc-obj1-bleed-dates}
\end{figure}

The average titre against circulating strains was calculated separately for cell and egg viruses as well as for each site. The procedure was as follows. For each individual, calculate mean log titre per clade (only relevant for 3c1 as it is the only clade with more than one virus of each type). Then for each individual, calculate weighted (by clade frequency) mean of clade-specific mean log titres. This yields one observation per individual which can be interpreted as the mean titre against the circulating strains weighted by their frequencies (Figure \ref{fig:cdc-obj1-ind-av-circulating}). Then for each vaccination history group calculate the mean log titre and its 95\% CI (normal approximation CI for the mean) which can be found in Figure \ref{fig:cdc-obj1-gmt-circulating}.

\begin{figure}
	\includegraphics[width=\textwidth,height=\textheight,keepaspectratio]{../data-summary/cdc-obj1-ind-av-circulating.pdf}
	\caption{Objective 1 weighted average titres agains circulating strains for each individual. Split by vaccination history, egg/cell viruses, site, year and prior vaccination count. Colored by individual ID.  The percentage in the top-left corner shows the proportion of circulating flu (as per Nexstrain) accounted for by the strains that were avaraged.}
	\label{fig:cdc-obj1-ind-av-circulating}
\end{figure}

\begin{figure}
	\includegraphics[width=\textwidth,height=\textheight,keepaspectratio]{../data-summary/cdc-obj1-gmt-circulating.pdf}
	\caption{Objective 1 means of weighted average titres agains circulating strains for each vaccination history group. Split by egg/cell and site.}
	\label{fig:cdc-obj1-gmt-circulating}
\end{figure}

Note that we do not have egg/cell pairs for every clade and the clades that we have titre data for do not account for all of the circulating flu. There is also no accounting for the geographic region as its frequencies are assumed to be similar to the global Nexstrain ones.

The vaccinated groups shows higher titres pre-vaccination and lower titres post-vaccination for the Israel sample. The pattern is less clear for the Peru sample.

\subsection{Bilthoven viruses}

The Bilthoven viruses do not appear to respond to the vaccine in either group.
They are old viruses and it is possible that only the people who were alive when
these viruses circulated (i.e. born pre-1968) would have titres against them.

Figure \ref{fig:cdc-obj1-yobs} shows the distribution of years of birth by vaccination history for objective 1 participants. There are a few more older individuals in the  group with 5 prior vaccinations and a few more yonger individuals in the  group with 0 prior, but the overall age distribution is similar in all groups.

\begin{figure}
	\includegraphics[width=\textwidth,height=\textheight,keepaspectratio]{../data-summary/cdc-obj1-yobs.pdf}
	\caption{Distribution of years of birth by vaccination history for objective 1 participants.}
	\label{fig:cdc-obj1-yobs}
\end{figure}

Figure \ref{fig:cdc-obj1-ind-bilthoven} shows the average titre against the four Bilthoven viruses for
the individuals split into three age groups (born pre-1968, 1969-1980 and post-1980). Individuals with 5 prior vaccinations appear to have more heterogenous titres which may be caused by the fact that there is more of them in the data.

Figure \ref{fig:cdc-obj1-gmt-bilthoven} shows the GMTs (against the all four Bilthoven viruses that were averaged for each individual) for groups with different vaccination histories. The pattern of yonger individuals having lower titres is evident in all groups.

Age appears to be a stronger contributor to the Bilthoven titres than vaccination history.

\begin{figure}
	\includegraphics[width=\textwidth,height=\textheight,keepaspectratio]{../data-summary/cdc-obj1-ind-bilthoven.pdf}
	\caption{Objective 1 average titres against the bilthoven viruses for each individual. Numbers indicate the number of individuals in the corresponding groups.}
	\label{fig:cdc-obj1-ind-bilthoven}
\end{figure}

\begin{figure}
	\includegraphics[width=\textwidth,height=\textheight,keepaspectratio]{../data-summary/cdc-obj1-gmt-bilthoven.pdf}
	\caption{Objective 1 means of average titres against the bilthoven viruses for each vaccination history group. }
	\label{fig:cdc-obj1-gmt-bilthoven}
\end{figure}

\section{Objective 2}

Investigate HI antibody titres for infrequently vaccinated (at most 3 times in the past 5 years) over 3 years of the study where
they are vaccinated each each.

Hypothesis: post-vaccination titres and the pre/post vaccination titre rise
decreases each year.

\subsection{Data}

Table \ref{tab:cdc-participant-summary-obj2} contains sex, age and
vaccination history summaries.
The majority of subjects are
women and were around 46 years of age when the study started (2016).
Similar numbers of people were vaccinated 1, 2 or 3 times in the years before
the study.

\input{../data-summary/cdc-participant-summary-obj2.tex}

\subsection{Titres}

Figure \ref{fig:cdc-obj2-gmts-1} shows GMTs for each virus at each timepoint for each year for the two sites.
Some viruses (e.g. Bilthoven) do not show much change across the timepoints in
any year. Some viruses (e.g. A/Switzerland/8060/17e) consistently show the pattern of post-vaccination titres being the highest and post-season titres being between pre and post vaccination ones.

\begin{figure}
	\includegraphics[width=\textwidth,height=\textheight,keepaspectratio]{../data-summary/cdc-obj2-gmts-1.pdf}
	\caption{Objective 2 geometric mean titres and 95\% confidence intervals for each virus at each timepoint for each study year for each site. Faceted by study year and colored by timepoint. The shaded region indicates the vaccine strain.}
	\label{fig:cdc-obj2-gmts-1}
\end{figure}

Figure \ref{fig:cdc-obj2-gmts-2} shows GMTs for each virus at each timepoint for each year for the two sites. This is similar to Figure \ref{fig:cdc-obj2-gmts-1} but layed out such that the same timepoint can be easily compared across years. The timepoint titres appear fairly similar across years for the majority of viruses. For some viruses (e.g., A/Brisbane/10/07p), the titres appear to increase in subsequent years as compared to previous years for every timepoint. For others (e.g. A/Thailand/409/05) they appear to decrease for some timepoints for at least one site.

\begin{figure}
	\includegraphics[width=\textwidth,height=\textheight,keepaspectratio]{../data-summary/cdc-obj2-gmts-2.pdf}
	\caption{Objective 2 geometric mean titres and 95\% confidence intervals for each virus at each timepoint for each study year. Faceted by timepoint and colored by study year. The shaded region indicates the vaccine strain (study year 1 and 2 - A/Hong Kong/4801/14e, study year 3 - A/Singapore/16-0019/16e).}
	\label{fig:cdc-obj2-gmts-2}
\end{figure}

\subsection{Vaccine responses}

Figure \ref{fig:cdc-obj2-vax-resp-virus} shows vaccine responses (ratio of post and pre vaccination titres)
for all participants across the 3 years of the study for each virus for the Israel and Peru samples.

The majority of viruses in the Israel sample show the pattern of decreasing response over the 3 years. This is less consistent in the Peru sample. For some viruses
there is a pattern of decreasing response (e.g., A/Singapore/16-0019/16e), for some the response stays similar (e.g., A/Netherlands/178/95), for some it increases
(e.g., A/Sydney/22/18), for some the year 1 and 3 show similar responses while
year 2 shows a lower response (e.g., A/Kansas/14/17e).

\begin{figure}
	\includegraphics[width=\textwidth,height=\textheight,keepaspectratio]{../data-summary/cdc-obj2-vax-resp-virus.pdf}
	\caption{Objective 2 ratios of post-vaccination titres to pre-vaccination titres and 95\% confidence intervals for each virus for each study year for the Israel and Peru samples. The shaded region indicates the vaccine strain (study year 1 and 2 - A/Hong Kong/4801/14e, study year 3 - A/Singapore/16-0019/16e).}
	\label{fig:cdc-obj2-vax-resp-virus}
\end{figure}

\subsection{Circulating strains}

The list of viruses for each clade is in Table \ref{tab:cdc-clade-viruses}. Frequencies of different virus clades during the study period are in Figure \ref{fig:cdc-clade-freq}. The timing of sample collection is in Figure \ref{fig:cdc-obj2-bleed-dates}. The most relevant set of frequencies is the first halves of 2017-2019 for Israel and the second halves of 2016-2018 for Peru as they appear to be the influenza "seasons" that the titres belong to.

\begin{figure}
	\includegraphics[width=\textwidth,height=\textheight,keepaspectratio]{../data-summary/cdc-obj2-bleed-dates.pdf}
	\caption{Bleed dates for objective 2 data. Each horizontal line represents a participant. Points are bleeds.}
	\label{fig:cdc-obj2-bleed-dates}
\end{figure}

The procedure for calculating the "GMT against circulating strains" was the same as in objective 1. The averages for each individual are in Figure \ref{fig:cdc-obj2-ind-circulating}. The yearly GMTs are in Figure \ref{fig:cdc-obj2-gmt-circulating}. The Israel site shows decreasing titres in subsequent years for all timepoints except pre-vaccination for egg viruses where there is an increase in year 2. The Peru site shows a decrease in year 2 and a return to almost year 1 levels in year 3 (especially pronounced for cell viruses).

\begin{figure}
	\includegraphics[width=\textwidth,height=\textheight,keepaspectratio]{../data-summary/cdc-obj2-ind-circulating.pdf}
	\caption{Weighted average titres agains circulating strains for each individual. Split by site and egg/cell viruses. Colored by individual ID.  The percentage in the top-left corner shows the proportion of circulating flu (as per Nexstrain) accounted for by the strains that were avaraged.}
	\label{fig:cdc-obj2-ind-circulating}
\end{figure}

\begin{figure}
	\includegraphics[width=\textwidth,height=\textheight,keepaspectratio]{../data-summary/cdc-obj2-gmt-circulating.pdf}
	\caption{Means of weighted average titres agains circulating strains for each year. Split by egg/cell and site.}
	\label{fig:cdc-obj2-gmt-circulating}
\end{figure}

\end{document}
