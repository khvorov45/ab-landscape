\documentclass[12pt]{article}

\usepackage{float}

\usepackage{standalone}

%\usepackage[utf8x]{inputenc}

%%% PAGE DIMENSIONS
\usepackage{geometry}
\geometry{a4paper}
\geometry{margin=2.54cm} % for example, change the margins to 2 inches all round

\usepackage{graphicx} % support the \includegraphics command and options

\usepackage[parfill]{parskip} % Activate to begin paragraphs with an empty line rather than an indent

%%% PACKAGES
\usepackage{booktabs} % for much better looking tables
\usepackage{array} % for better arrays (e.g., matrices) in maths
\usepackage{paralist} % very flexible & customisable lists (e.g., enumerate/itemise, etc.)
\usepackage{verbatim} % adds environment for commenting out blocks of text & for better verbatim
\usepackage{subfig} % make it possible to include more than one captioned figure/table in a single float
% These packages are all incorporated in the memoir class to one degree or another...

\usepackage{multicol}
\usepackage{multirow}
\usepackage{xcolor}
\usepackage{amsmath}

\usepackage[T1]{fontenc}
\usepackage{lmodern}

% Sans-serif font
\renewcommand{\familydefault}{\sfdefault}

\usepackage{makecell}

\renewcommand{\arraystretch}{1.1}

%%% HEADERS & FOOTERS
\usepackage{fancyhdr} % This should be set AFTER setting up the page geometry
\pagestyle{fancy} % options: empty , plain , fancy
\fancyhead[L]{\leftmark}
\fancyhead[C]{}
\fancyhead[R]{\rightmark}
\fancyfoot[L]{}
\fancyfoot[C]{}
\fancyfoot[R]{\thepage}
\renewcommand{\headrulewidth}{0pt}
\renewcommand{\footrulewidth}{0pt}
\setlength{\headheight}{52pt}

\fancypagestyle{plain}{
	\fancyhf{} % clear all header and footer fields
	\fancyfoot[R]{\thepage} % except the center
	\renewcommand{\headrulewidth}{0pt}
	\renewcommand{\footrulewidth}{0pt}
}

%%% BIBLIOGRAPHY
\usepackage[numbers]{natbib}
\bibliographystyle{vancouver}

%%% SECTION TITLE APPEARANCE
\usepackage{sectsty}
\allsectionsfont{\sffamily\mdseries\upshape}

%%% ToC (table of contents) APPEARANCE
%\usepackage[nottoc,notlof,notlot]{tocbibind} % Put the bibliography in the ToC
%\usepackage[titles,subfigure]{tocloft} % Alter the style of the Table of Contents
%\renewcommand{\cftsecfont}{\rmfamily\mdseries\upshape}
%\renewcommand{\cftsecpagefont}{\rmfamily\mdseries\upshape} % No bold!

\usepackage[bookmarks,bookmarksnumbered,bookmarksopen,hidelinks]{hyperref}

\usepackage{bookmark}

% TITLE
\title{CDC}
\author{Arseniy Khvorov}
\begin{document}

\maketitle

\section{Objective 1}

Compare post-vaccination HI antibody titres for frequently vaccinated (in the past) vs infrequently vaccinated.

Hypothesis: post-vaccination titres and the pre/post vaccination titre rise
is lower in the frequently vaccinated.

\subsection{Data}

All data analysed for objective 1 comes from the Israel study sample.

Table \ref{tab:cdc-participant-summary-obj1} contains sex and age summaries.
The majority of subjects in both the frequent and the infrequent group are
women and were around 40 years of age when the study started (2016).

\input{../data-summary/cdc-participant-summary-obj1.tex}

\subsection{HI titres}

Geometric mean titres for each virus at each timepoint for each group (frequently
vaccinated vs infrequently vaccinated) are shown in Figure
\ref{fig:cdc-obj1-timepoint-gmts}.
Pre-vaccination GMTs are generally close to the same
or higher for all viruses for the frequently vaccinated.
Post-vaccination GMTs are generally close to the same or higher in the infrequently vaccinated with
the exception of the Bilthoven viruses (which show the same pattern
of frequently vaccinated GMTs being higher at every
timepoint). The pattern of GMTs for the post-season timepoint is close to that
of post-vaccination except that titres are generally more similar between the
frequently and infrequently vaccinated.

\begin{figure}
	\includegraphics[width=\textwidth,height=\textheight,keepaspectratio]{../data-summary/cdc-obj1-timepoint-gmts.pdf}
	\caption{Geometric mean titres and 95\% confidence intervals for each virus at each timepoint for the frequently and the infrequently vaccinated. The shaded region indicates the vaccine strain.}
	\label{fig:cdc-obj1-timepoint-gmts}
\end{figure}

\subsection{HI titre changes}

Ratio of post-vaccination titres to pre-vaccination titres is shown in Figure
\ref{fig:cdc-obj1-timepoint-diffs}. Ratios are higher for the infrequently
vaccinated for the majority of viruses.

\begin{figure}
	\includegraphics[width=\textwidth,height=\textheight,keepaspectratio]{../data-summary/cdc-obj1-timepoint-diffs.pdf}
	\caption{Ratios of post-vaccination titres to pre-vaccination titres and 95\% confidence intervals for each virus for the frequently and the infrequently vaccinated. The shaded region indicates the vaccine strain.}
	\label{fig:cdc-obj1-timepoint-diffs}
\end{figure}

\subsection{GMT against circulating strains}

Frequencies of different virus clades during the study period are in Figure \ref{fig:cdc-obj1-clade-freq}. The timing of sample collection is in Figure \ref{fig:cdc-obj1-bleed-dates}. The most relevant set of frequencies is the first half of 2019 as it appers to be the influenza "season" that the titres belong to. The list of viruses for each clade is in Table \ref{tab:cdc-obj1-clade-viruses}.

\begin{figure}
	\includegraphics[width=\textwidth,height=\textheight,keepaspectratio]{../data-summary/cdc-obj1-clade-freq.pdf}
	\caption{Clade frequencies from 2016 to 2019. Two sets of frequencies per year. Data from Nexstrain. Whenever they do not add up to 100\%, it means that the remaining circulating flu was made up of clades whose viruses we do not have titre data against (e.g., clade 'A3')}
	\label{fig:cdc-obj1-clade-freq}
\end{figure}

\begin{figure}
	\includegraphics[width=\textwidth,height=\textheight,keepaspectratio]{../data-summary/cdc-obj1-bleed-dates.pdf}
	\caption{Bleed dates for objective 1 data. Each horizontal line represents a participant. Points are bleeds.}
	\label{fig:cdc-obj1-bleed-dates}
\end{figure}

\input{../data-summary/cdc-obj1-clade-viruses.tex}

The average titre against circulating strains was calculated separately for cell and egg viruses. The procedure was as follows. For each individual, calculate mean log titre per clade (only relevant for 3c1 as it is the only clade with more than one virus of each type). Then for each individual, calculate weighted (by clade frequency) mean of clade-specific mean log titres. This yields one observation per individual which can be interpreted as the mean titre against the circulating strains weighted by their frequencies (Figure \ref{fig:cdc-obj1-ind-av-circulating}). Then for each group (frequent and infrequent) calculate the mean log titre and its 95\% CI (normal approximation CI for the mean) which can be found in Figure INSERT.

\begin{figure}
	\includegraphics[width=\textwidth,height=\textheight,keepaspectratio]{../data-summary/cdc-obj1-ind-av-circulating.pdf}
	\caption{Weighted average titres agains circulating strains for each individual. Split by group and egg/cell viruses. Colored by individual ID.}
	\label{fig:cdc-obj1-ind-av-circulating}
\end{figure}

\begin{figure}
	\includegraphics[width=\textwidth,height=\textheight,keepaspectratio]{../data-summary/cdc-obj1-gmt-circulating.pdf}
	\caption{Means of weighted average titres agains circulating strains for each group. Split by egg/cell.}
	\label{fig:cdc-obj1-gmt-circulating}
\end{figure}

Note that we do not have egg/cell pairs for every clade and the clades that we have titre data for do not account for all of the circulating flu. There is also no accounting for the geographic region as its frequencies are assumed to be similar to the global Nexstrain ones.

The frequent group shows higher titres pre-vaccination (especially for the egg viruses) and lower titres post-vaccination (especially for the cell viruses) and well as slightly lower post-season titres for the cell viruses.

\subsection{Interpretation}

The vaccine response (titre change between post and pre vaccination timepoints)
appears to be lower in the frequently vaccinated. However, the actual
post-vaccination titres between the two groups are fairly close for most viruses.

Both groups had low titres against A/South Australia/34/2019 and A/Sydney/22/2018 pre-vaccination. Those who were infrequently vaccinated had those titres rise to above 20 (on average) while the frequently vaccinated titres stayed below 20 (on average). This indicates that the depressed vaccine response mey be most evident for viruses that individuals have no/low titres against before vaccination (which could be novel strains, for example).

Neither group had a strong response to the Bilthoven viruses. However, the frequently vaccinated group had higher titres against them throughout the season. This is likely due to the fact that while the current vaccine did not generate antibody production against these viruses, one of the previous vaccines did and the frequently vaccinated group retained those antibodies. This indicates that the frequently vaccinated are likely to be protected more broadly due the variety of vaccines they received in the past and the longevity of antibody titres.

\section{Objective 2}

Investigate HI antibody titres for infrequently vaccinated over 3 years where
they are vaccinated each each.

Hypothesis: post-vaccination titres and the pre/post vaccination titre rise
decreases each year.

\subsection{Data}

Table \ref{tab:cdc-participant-summary-obj2} contains sex, age and
vaccination history summaries.
The majority of subjects are
women and were around 46 years of age when the study started (2016).
Similar numbers of people were vaccinated 1, 2 or 3 times in the years before
the study.

\input{../data-summary/cdc-participant-summary-obj2.tex}

\subsection{Titres}

Figure \ref{fig:cdc-obj2-gmts-1-israel} shows GMTs for each virus at each timepoint for each year for the Israel sample. Figure \ref{fig:cdc-obj2-gmts-1-peru} shows the same for the Peru sample.
Some viruses (e.g. Bilthoven) do not show much change across the timepoints in
any year. Some viruses (e.g. A/Switzerland/8060/17e) consistently show the pattern of post-vaccination titres being the highest and post-season titres being between pre and post vaccination ones.

\begin{figure}
	\includegraphics[width=\textwidth,height=\textheight,keepaspectratio]{../data-summary/cdc-obj2-gmts-1-israel.pdf}
	\caption{Israel sample. Geometric mean titres and 95\% confidence intervals for each virus at each timepoint for each study year. Faceted by study year and colored by timepoint. The shaded region indicates the vaccine strain.}
	\label{fig:cdc-obj2-gmts-1-israel}
\end{figure}

\begin{figure}
	\includegraphics[width=\textwidth,height=\textheight,keepaspectratio]{../data-summary/cdc-obj2-gmts-1-peru.pdf}
	\caption{Peru sample. Geometric mean titres and 95\% confidence intervals for each virus at each timepoint for each study year. Faceted by study year and colored by timepoint. The shaded region indicates the vaccine strain.}
	\label{fig:cdc-obj2-gmts-1-peru}
\end{figure}

Figure \ref{fig:cdc-obj2-gmts-2-israel} shows GMTs for each virus at each timepoint for each year for the Israel sample. Figure \ref{fig:cdc-obj2-gmts-2-peru} shows the same for the Peru sample. These are similar to Figures \ref{fig:cdc-obj2-gmts-1-israel} and \ref{fig:cdc-obj2-gmts-1-peru} but layed out such that the same timepoint can be easily compared across years. The timepoint titres appear fairly similar across years. For some viruses (e.g., A/Brisbane/10/07p), the titres appear to increase in subsequent years as compared to previous years for every timepoint.

\begin{figure}
	\includegraphics[width=\textwidth,height=\textheight,keepaspectratio]{../data-summary/cdc-obj2-gmts-2-israel.pdf}
	\caption{Israel sample. Geometric mean titres and 95\% confidence intervals for each virus at each timepoint for each study year. Faceted by timepoint and colored by study year. The shaded region indicates the vaccine strain (study year 1 and 2 - A/Hong Kong/4801/14e, study year 3 - A/Singapore/16-0019/16e).}
	\label{fig:cdc-obj2-gmts-2-israel}
\end{figure}

\begin{figure}
	\includegraphics[width=\textwidth,height=\textheight,keepaspectratio]{../data-summary/cdc-obj2-gmts-2-peru.pdf}
	\caption{Peru sample. Geometric mean titres and 95\% confidence intervals for each virus at each timepoint for each study year. Faceted by timepoint and colored by study year. The shaded region indicates the vaccine strain (study year 1 and 2 - A/Hong Kong/4801/14e, study year 3 - A/Singapore/16-0019/16e).}
	\label{fig:cdc-obj2-gmts-2-peru}
\end{figure}

\subsection{Vaccine responses}

Figure \ref{fig:cdc-obj2-vax-resp-virus} shows vaccine responses (ratio of post and pre vaccination titres)
for all participants across the 3 years of the study for each virus for the Israel and Peru samples. Vaccine
response does not follow a consistent pattern across 3 years for all viruses. For some viruses
there is a pattern of decreasing response (e.g., A/Singapore/16-0019/16e), for some the response stays similar (e.g., A/Netherlands/178/95), for some it increases
(e.g., A/Sydney/22/18), for some the year 1 and 3 show similar responses while
year 2 shows a lower response (e.g., A/Kansas/14/17e).

\begin{figure}
	\includegraphics[width=\textwidth,height=\textheight,keepaspectratio]{../data-summary/cdc-obj2-vax-resp-virus.pdf}
	\caption{Ratios of post-vaccination titres to pre-vaccination titres and 95\% confidence intervals for each virus for each study year for the Israel and Peru samples. The shaded region indicates the vaccine strain (study year 1 and 2 - A/Hong Kong/4801/14e, study year 3 - A/Singapore/16-0019/16e).}
	\label{fig:cdc-obj2-vax-resp-virus}
\end{figure}

Figure \ref{fig:cdc-obj2-average-response} shows average responses for each individual across the 3 years (i.e., for each individual, the ratios of pre and post-vaccination titres were averaged across viruses for each year).
The pattern is inconsistent with some individuals showing a decreasing response,
some - increasing and some with year 2 being either higher of lower
than years 1 and 3.
The egg viruses show higher responses for some individuals.

\begin{figure}
	\includegraphics[width=\textwidth,height=\textheight,keepaspectratio]{../data-summary/cdc-obj2-average-response.pdf}
	\caption{Averaged (across viruses) responses (post/pre vaccination titre ratios) for each individual at each study year. Split by site. Responses were averaged across cell and egg viruses separately. Line colors correspond to participant IDs.}
	\label{fig:cdc-obj2-average-response}
\end{figure}

\subsection{Interpretation}

There is no clear trend of a decreasing vaccine response on the whole serology
panel. Nor is there a clear trend of post-vaccianation titres being lower in subsequent years as compared to previous years. This runs somewhat contrary to
the data for objective 1. For example, the post-vaccination titres in Figure \ref{fig:cdc-obj1-timepoint-gmts} are generally higher for the infrequently vaccinated but the post-vaccination titres in Figures \ref{fig:cdc-obj2-gmts-2-israel} and \ref{fig:cdc-obj2-gmts-2-peru} do not show the equivalent pattern of decresing in subsequent years as compared to previous years (i.e. as subjects become more highly vaccinated). There are examples of the opposite pattern (e.g., A/Sydney/22/18), where the post-vaccination GMTs appear to increase as subjects become more highly vaccinated.

\end{document}
