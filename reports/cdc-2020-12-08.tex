\documentclass[12pt]{article}

\usepackage{float}

\usepackage{standalone}

%\usepackage[utf8x]{inputenc}

%%% PAGE DIMENSIONS
\usepackage{geometry}
\geometry{a4paper}
\geometry{margin=2.54cm} % for example, change the margins to 2 inches all round

\usepackage{graphicx} % support the \includegraphics command and options

\usepackage[parfill]{parskip} % Activate to begin paragraphs with an empty line rather than an indent

%%% PACKAGES
\usepackage{booktabs} % for much better looking tables
\usepackage{array} % for better arrays (e.g., matrices) in maths
\usepackage{paralist} % very flexible & customisable lists (e.g., enumerate/itemise, etc.)
\usepackage{verbatim} % adds environment for commenting out blocks of text & for better verbatim
\usepackage{subfig} % make it possible to include more than one captioned figure/table in a single float
% These packages are all incorporated in the memoir class to one degree or another...

\usepackage{multicol}
\usepackage{multirow}
\usepackage{xcolor}
\usepackage{amsmath}

\usepackage[T1]{fontenc}
\usepackage{lmodern}

% Sans-serif font
\renewcommand{\familydefault}{\sfdefault}

\usepackage{makecell}

\renewcommand{\arraystretch}{1.1}

%%% HEADERS & FOOTERS
\usepackage{fancyhdr} % This should be set AFTER setting up the page geometry
\pagestyle{fancy} % options: empty , plain , fancy
\fancyhead[L]{\leftmark}
\fancyhead[C]{}
\fancyhead[R]{\rightmark}
\fancyfoot[L]{}
\fancyfoot[C]{}
\fancyfoot[R]{\thepage}
\renewcommand{\headrulewidth}{0pt}
\renewcommand{\footrulewidth}{0pt}
\setlength{\headheight}{52pt}

\fancypagestyle{plain}{
	\fancyhf{} % clear all header and footer fields
	\fancyfoot[R]{\thepage} % except the center
	\renewcommand{\headrulewidth}{0pt}
	\renewcommand{\footrulewidth}{0pt}
}

%%% BIBLIOGRAPHY
\usepackage[numbers]{natbib}
\bibliographystyle{vancouver}

%%% SECTION TITLE APPEARANCE
\usepackage{sectsty}
\allsectionsfont{\sffamily\mdseries\upshape}

%%% ToC (table of contents) APPEARANCE
%\usepackage[nottoc,notlof,notlot]{tocbibind} % Put the bibliography in the ToC
%\usepackage[titles,subfigure]{tocloft} % Alter the style of the Table of Contents
%\renewcommand{\cftsecfont}{\rmfamily\mdseries\upshape}
%\renewcommand{\cftsecpagefont}{\rmfamily\mdseries\upshape} % No bold!

\usepackage[bookmarks,bookmarksnumbered,bookmarksopen,hidelinks]{hyperref}

\usepackage{bookmark}

% TITLE
\title{CDC}
\author{Arseniy Khvorov}
\begin{document}

\maketitle

\section{Objective 1}

Compare post-vaccination HI antibody titres for frequently vaccinated (in the past) vs infrequently vaccinated.

Hypothesis: post-vaccination titres and the pre/post vaccination titre rise
is lower in the frequently vaccinated.

\subsection{Data}

Table \ref{tab:cdc-participant-summary} contains sex and age summaries.
The majority of subjects in both the frequent and the infrequent group are
women and were around 40 years of age when the study started (2016).

\input{../data-summary/cdc-participant-summary.tex}

\subsection{HI titres}

Geometric mean titres for each virus at each timepoint for each group (frequently
vaccinated vs infrequently vaccinated) are shown in Figure
\ref{fig:cdc-obj1-timepoint-gmts}.
Pre-vaccination GMTs are generally close to the same
or higher for all viruses for the frequently vaccinated.
Post-vaccination GMTs are generally close to the same or higher in the infrequently vaccinated with
the exception of the Bilthoven viruses (which show the same pattern
of frequently vaccinated GMTs being higher at every
timepoint). The pattern of GMTs for the post-season timepoint is close to that
of post-vaccination except that titres are generally more similar between the
frequently and infrequently vaccinated.

\begin{figure}
	\includegraphics[width=\textwidth,height=\textheight,keepaspectratio]{../data-summary/cdc-obj1-timepoint-gmts.pdf}
	\caption{Geometric mean titres and 95\% confidence intervals for each virus at each timepoint for the frequently and the infrequently vaccinated.}
	\label{fig:cdc-obj1-timepoint-gmts}
\end{figure}

\subsection{HI titre changes}

Ratio of post-vaccination titres to pre-vaccination titres is shown in Figure
\ref{fig:cdc-obj1-timepoint-diffs}. Ratios are higher for the infrequently
vaccinated for the majority of viruses.

\begin{figure}
	\includegraphics[width=\textwidth,height=\textheight,keepaspectratio]{../data-summary/cdc-obj1-timepoint-diffs.pdf}
	\caption{Ratios of post-vaccination titres to pre-vaccination titres and 95\% confidence intervals for each virus at each timepoint for the frequently and the infrequently vaccinated.}
	\label{fig:cdc-obj1-timepoint-diffs}
\end{figure}

\subsection{Interpretation}

The vaccine response (titre change between post and pre vaccination timepoints)
appears to be lower in the frequently vaccinated. However, the actual
post-vaccination titres between the two groups are fairly close for most viruses.

Both groups had low titres against A/South Australia/34/2019 and A/Sydney/22/2018 pre-vaccination. Those who were infrequently vaccinated had those titres rise to above 20 (on average) while the frequently vaccinated titres stayed below 20 (on average). This indicates that the depressed vaccine response mey be most evident for viruses that individuals have no/low titres against before vaccination (which could be novel strains, for example).

Neither group had a strong response to the Bilthoven viruses. However, the frequently vaccinated group had higher titres against them throughout the season. This is likely due to the fact that while the current vaccine did not generate antibody production against these viruses, one of the previous vaccines did and the frequently vaccinated group retained those antibodies. This indicates that the frequently vaccinated are likely to be protected more broadly due the variety of vaccines they received in the past and the longevity of antibody titres.

\section{Objective 2}

Investigate HI antibody titres for infrequently vaccinated over 3 years where
they are vaccinated each each.

Hypothesis: post-vaccination titres and the pre/post vaccination titre rise
decreases each year.

\subsection{Data}

Participants were vaccinated 1-3 times in the 5 years before the study
(2011-2015).

\end{document}
